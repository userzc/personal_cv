\documentclass[11pt,letter]{moderncv}
% \moderncvstyle{casual}       % idem
% \moderncvcolor{orange}
% \moderncvtheme[orange]{casual}
\moderncvtheme[burgundy]{banking}
% optional argument are 'nocolor' (black & white cv) and 'roman' (for
% roman fonts, instead of sans serif fonts)
% \moderncvstyle{casual}

% character encoding
\usepackage[utf8]{inputenc}
\usepackage[T1]{fontenc}
\usepackage{lmodern}
\usepackage{graphicx}
\usepackage{geometry}
\usepackage{textcomp}
% \usepackage{marvosym}           % símbolos especiales, parece no ocuparse
\usepackage[]{comment}


% personal data (the given example is exhaustive; just give what you want)
\firstname{\huge René}
\familyname{\huge Zurita Corro}
\title{Licenciado en Matemáticas Aplicadas}
\address{Calle Micaela Galindo, 48}{Huajuapan de León, Oaxaca}  % for classic style
\phone{(953)~53~22954}
\mobile{(953)~111~7761}
\email{zc.rene@gmail.com}
\social[github]{userzc}

\extrainfo{Ced. Prof.: 5986352 -- R.F.C.: ZUCR8510151J4}

% \extrainfo{R.F.C.: ZUCR8510151J4 \\
%   Ced. Prof.: 5986352}

\photo[64pt]{FotoFrente.eps} % also optional, and the optional
% argument is the height the picture must
% be resized to

% \quote{Any intelligent fool can make things bigger, more complex, and
% more violent. It takes a touch of genius -- and a lot of courage --
% to move in the opposite direction.}% also optional

\begin{document}

\makecvtitle

\section{Formación Académica}
\label{sec:formacion-academica}
\cventry{2009 -- 2011}{Maestría}{Centro de Investigación en
  Matemáticas (CIMAT)}{Guanajuato, Gto.}{\textit{Maestría en
    Matemáticas Aplicadas}}{Titulación en proceso.}

\cventry{2003 -- 2008}{Licenciatura}{Universidad Tecnológica de la
  Mixteca (UTM)}{Huajuapan de León, Oax.}{\textit{Licenciatura en
    Matemáticas Aplicadas}}{}

\section{Experiencia}
\label{sec:experiencia}
\subsection{Profesional}
\label{sec:profesional}



\cventry{Ene 2019 -- Presente}{Desarrollador de Software}{Veureka
  FVP}{Huajuapan de León, Oax.}{} {Análisis de datos y sysadmin, \emph{Firebase}, \emph{pandas}.}

\cventry{Ene 2018 -- Dic 2018}{Desarrollador de Software}{Veureka
  FVP}{Huajuapan de León, Oax.}{} {Prueba de concepto para transacciones y pagos utilizando monedas fiat y criptomonedas.}


\cventry{Ene 2017 -- Dic 2017}{Desarrollador de Software}{Veureka
  FVP}{Huajuapan de León, Oax.}{} {Redes públicas y privadas con protocolos de blockchain, \emph{Ripple} y \emph{Stellar}.}

\cventry{Feb 2014 -- Dic 2016}{Desarrollador de Software}{Veureka
  FVP}{Huajuapan de León, Oax.}{} {Desarrollo de proyectos de software
  con componentes de optimización {\em Java}.}

\cventry{2014}{Asesor Externo}{Colegio de Bachilleres del Estado de
  Oaxaca}{Oaxaca de Juárez, Oaxaca}{} {Impartición de asesorías
  especializadas de Matemáticas para preparación al representante del
  {\em COBAO} en el {\em XXIII Encuentro Académico, Cultural y
    Deportivo de la Zona Sur-Sureste}, Primer Lugar obtenido.}

\cventry{2013}{Asesor Externo}{Colegio de Bachilleres del Estado de
  Oaxaca}{Oaxaca de Juárez, Oaxaca}{} {Impartición de asesorías
  especializadas de Matemáticas para preparación al representante del
  {\em COBAO} en el {\em XXII Encuentro Académico, Cultural y
    Deportivo de la Zona Sur-Sureste}, Primer Lugar obtenido.}

\cventry{2012}{Asesor Externo}{Colegio de Bachilleres del Estado de
  Oaxaca}{Oaxaca de Juárez, Oaxaca}{} {Impartición de asesorías
  especializadas de Matemáticas para preparación al representante del
  {\em COBAO} en el {\em XXI Encuentro Académico, Cultural y Deportivo
    de la Zona Sur-Sureste}, Primer Lugar obtenido.}

\cventry{2008 -- 2009}{Catedrático}{Instituto Mixteco de Educación
  Superior}{Tezoatlán de Segura y Luna, Oaxaca}{}{Materias
  impartidas:\newline{}%
  % BEGIN RECEIVE ORGTBL ClasesImpartidas
  % END RECEIVE ORGTBL ClasesImpartidas
  \begin{itemize}%
  \item Informática I (64 horas clase).
  \item Informática II (64 horas clase).
  \item Matemáticas Básicas (80 horas clase).
  \item Matemáticas I (80 horas clase).
  \item Matemáticas Financieras (80 horas clase).
  \item Matemáticas II (80 horas clase).
  \end{itemize}
}

% Por alguna razon no me permite utilizar las tablas de manera
% adecuada, por lo que no activo la siguiente tabla, quizá
% en otra versión
\begin{comment}
  #+ORGTBL: SEND ClasesImpartidas orgtbl-to-latex :splice nil :skip 0
  | Materia                 | Horas-Clase |
  |-------------------------+-------------|
  | Informática I           |          64 |
  | Informática II          |          64 |
  | Matemáticas Básicas     |          80 |
  | Matemáticas I           |          80 |
  | Matemáticas Financieras |          80 |
  | Matemáticas II          |          80 |
\end{comment}


\cventry{2008 -- 2009}{Coordinador Académico}{Instituto Mixteco de
  Educación Superior}{Tezoatlán de Segura y Luna,
  Oaxaca}{}{Responsabilidades:\newline{}%
  \begin{itemize}%
  \item Estadísticas de calificaciones globales.
  \item Elaboración de horario escolar.
  \end{itemize}
}

\subsection{Otros cursos y distinciones}
\label{sec:otros-cursos}

\cventry{2009}{Segundo foro nacional de tesis de licenciatura en
  matemáticas}{}{}{}{Participación en el evento realizado del 17 al 19
  de junio.}

\cventry{2009}{Segunda escuela internacional de modelación matemática
  y sus aplicaciones}{Sobre la generación de superficies ópticas con
  herramientas de pulido clásico}{}{}{Participación en sesión de
  carteles realizada del 12 al 16 de enero.}

\cventry{2008}{Titulación de tesis}{}{}{}{Aprobación por unanimidad.}

\cventry{2008}{Universidad Tecnológica de la Mixteca}{Huajuapan de
  León}{}{}{Apoyo técnico al equipo de cómputo del laboratorio de
  matemáticas aplicadas por un total de 60 horas.}

\cventry{2007}{Universidad Tecnológica de la Mixteca}{Huajuapan de
  León}{}{}{Impartición de la materia {\em Problemas de matemáticas}
  por un total de 420 horas a:\newline{}%
  \begin{itemize}
  \item Curso propedéutico corto, agosto-septiembre.
  \item Curso propedéutico largo, marzo-julio.
  \end{itemize}
}
\cventry{2006}{Universidad Tecnológica de la Mixteca}{Huajuapan de
  León}{}{}{Curso de \LaTeX, 2 horas diarias por 2 semanas.}


\cventry{2005}{Universidad Tecnológica de la Mixteca}{Huajuapan de
  León}{}{}{Curso de programación para el concurso {\em ACM-ICPC}, 8
  horas diarias por 2 meses.}


\section{Idiomas}
\label{sec:idiomas}
\cvlanguage{Español}{Nivel Alto}{Idioma materno.}
\cvlanguage{Inglés}{Nivel Alto}{Por parte de {\em UTM} y {\em CIMAT}.}

\section{Lenguajes de programación}
\label{sec:leng-prog}


\cvcomputer{General}{Python}{General}{node}
\cvcomputer{General}{C++}{General}{Java}
\cvcomputer{Scripting}{bash}{Scripting}{zsh}
\cvcomputer{Programación lógica}{LISP}{Programación lógica}{Prolog}
\cvcomputer{Optimización}{Optaplaner}{Análisis de datos}{pandas}
\cvcomputer{SaaS}{Google Cloud Platfom (GCP)}{SaaS}{Firebase}

\section{Conocimientos de computación}
\label{sec:conoc-de-comp}

\cvdoubleitem{SO}{Windows}{SO}{Linux (Ubuntu, Debian, etc.)}

\cvdoubleitem{Soft. de productividad}{MS Office, OpenOffice,
  LibreOffice}{Paquetes matemáticos}{Octave, MatLab, R, Maxima,
  Mathematica}

\cvdoubleitem{Internet}{Explorer, Firefox, Google Chrome, Google
  Chromium}{Software tipográfico}{\LaTeX, emacs, TeXMaker, TeXmacs}

\end{document}

%%% Local Variables:
%%% mode: latex
%%% TeX-master: t
%%% End:
